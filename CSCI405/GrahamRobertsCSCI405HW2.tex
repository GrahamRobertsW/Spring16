\documentclass[11pt]{article}
\usepackage{enumerate}
\usepackage{amsfonts}
\usepackage{graphicx}
\usepackage{amsmath}
\usepackage{mathtools}

\pagestyle{empty} \setlength{\parindent}{0mm}
\addtolength{\topmargin}{-0.5in} \setlength{\textheight}{8in}
\addtolength{\textwidth}{1in} \addtolength{\oddsidemargin}{-0.5in}

\begin{document}

\textbf{CSCI 405 Analysis of Algorithms II \hfill Spring 2016 - Fizzano}

\begin{center}
\textbf{Homework 2 \\  Due April 27}
\end{center}

\begin{enumerate}[1.]



\item Your friend Jason is working as a camp counselor and he is in charge of organizing a mini-triathalon on Friday the 13th for $n$ campers -- each contestant must swim 13 laps in the pool then bike 13 miles then run 1.3 miles.  The only issue is the pool is very small and Jason doesn't want to see anyone drown so he decides to do a staggered start to the race and let only one person at a time in the pool.  Thus, the first contestant swims 13 laps, gets out and starts biking.  As soon as the first person is done swimming the second person can start swimming and when he is done he starts biking and the third person can start swimming.  You get the idea.  

Because the campers have been training, Jason has a very good estimation of how long each camper will take to do each leg of the race.  Let $s_i, b_i, r_i$ be the projected time for camper $i$ to swim, bike and run respectively.  Jason wants to make an efficient schedule for the triathalon.  The schedule will dictate the order in which to start the contestants.   Let's say the {\em completion time} of the schedule is the earlist time at which all contestants have finished all three legs of the race.  Jason wants to minimize the completion time of the schedule.  

Note that participants can be biking and running simultaneously but at most one person can be in the pool at a time.  

Give an efficient algorithm that produces a schedule whose completion time is as small as possible (you have to assume that each contestant spends exactly the amount of time Jason estimates).  Prove your algorithm minimizes the completion time of the schedule.  Analyze the running time of your algorithm.  

The completion time of any set of runners $i$ whose resepective race times are $s_i, r_i, b_i$ respectively.  the total running time of the the race is the maximum of the set $T$, such that:

\begin{equation*}
T\equiv\left\{t_1,t_2,\ldots,t_n\right\}
\end{equation*}
Such that
\begin{equation*}
t_i=b_i+r_i+\sum_{j=1}^is_j
\end{equation*} 
Each runner has a finishing time equal to his/her running,biking, and swimming times, plus the swimming times of all previous racers.
 Because no two racers can be in pool simultaneously, a racer must wait for the previous racer to exit the pool before they may begin.
 The finishing time is once all racers have finished, so it is the maximum time in the set $T$. 
A racer may pass another during the running or biking phase if their running or biking time faster by a greater amount than he other persons head start.
The pool should alwyas be in use.
A racer should this is because the racetime is the maximum of set ($T$).  
For any racer $i$ whose completion time can be represented as $t_i+w_i$, where $w\ge0$ is some time spent waiting for Jason to clean the pool or something, the text 
\begin{equation*}
T_w={t_1+w_1, t_2+w_2,\ldots,t_n+w_n}
\end{equation*}
Since we know that every entry in $T_w$ is greater than or equal to the corresponding entry in $T$ the max of $T_w$ cannot be less than that of T.
 
The algorithm that minimizes this time is one in which the racers are sorted in ascending order by the sum of running and biking time.  That is the total time for that racer to finish AFTER they get out of the pool.
The racer who will take the longest to finish after leaving the pool will go first, and the racer with the quickest finishing time on the running and biking portion will go will go last.
Note this does not mean that the race time will be the sum of swim times plus the combined running and biking times of the faster runner/biker, although that outcome is not unlikely.
The proof that this is the fastest possible algorithm can be proven by comparing the finishing times of any two racers if you switch their positions.

\begin{align*}
\mathrm{let\,}t_i&\equiv\left(r_i+b_i+\sum_{j=0}^is_j\right)\\
\mathrm{let\,}T&\equiv\{t_1,t_2,\ldots,t_n\}:r_i+b_i\ge r_{i+1}+b_{i+1}\\
&\forall\mathrm{swap\,}(i,j):\\
T&=\{t_1,\ldots,t_{i-1},t_i,t_{i+1},\ldots,t_{j-1},t_j,t_{j+1},\ldots,t_n\}\\
T^\epsilon&\equiv\{t_1^\epsilon,\ldots,t_{i-1}^\epsilon,t_j^\epsilon,t_{i+1}^\epsilon,\ldots,t_{j-1}^\epsilon,t_i^\epsilon,t_{j+1}^\epsilon,\ldots,t_n^\epsilon\}\\
t_i&=\left(r_i+b_i+\sum_{\zeta=1}^{i-1}s_\zeta+s_i\right)\\
t_j&=\left(r_j+b_j+\sum_{\zeta=1}^{i-1}s_\zeta+s_i+\sum_{\zeta=i+1}^{j-1}s_\zeta+s_j\right)\\
t_i^\epsilon&=\left(r_i+b_i+\sum_{\zeta=1}^{i-1}s_\zeta+s_j+\sum_{\zeta=i+1}^{j-1}s_\zeta+s_i\right)\\
t_j^\epsilon&=\left(r_j+b_j+\sum_{\zeta=1}^{i-1}s_\zeta+s_j\right)\\
 \sum_{\zeta=1}^{i-1}s_\zeta+s_i&<\sum_{\zeta=1}^{i-1}s_\zeta+s_j+\sum_{\zeta=i+1}^{j-1}s_\zeta+s_i\\
&\implies r_i+b_i+\sum_{\zeta=1}^{i-1}s_\zeta+s_i<r_i+b_i+\sum_{\zeta=1}^{i-1}s_\zeta+s_j+\sum_{\zeta=i+1}^{j-1}s_\zeta+s_i\\
&\implies t_i<t_i^\epsilon\\
\sum_{\zeta=1}^{i-1}s_\zeta+s_i+\sum_{\zeta=i+1}^{j-1}s_\zeta+s_j&=\sum_{\zeta=1}^{i-1}s_\zeta+s_j+\sum_{\zeta=i+1}^{j-1}s_\zeta+s_i\\
S_j&\equiv\sum_{\zeta=1}^{i-1}s_\zeta+s_j+\sum_{\zeta=i+1}^{j-1}s_\zeta+s_i\\
r_i+b_i&\ge r_j+b_j\\
&\implies r_i+b_i+S_j\ge r_j+b_j+S_j\\
&\implies t_i^\epsilon\ge t_j\\
\end{align*}
\begin{align*}
&\forall t_k|i<k<j\\
 t_k&=r_k+b_k+\sum_{\zeta=1}^{i-1}s_\zeta+s_i+\sum_{\zeta=i+1}^ks_\zeta\\
S_j&=\sum_{\zeta=1}^{i<1}s_\zeta+s_i+\sum_{\zeta=1+1}^ks_\zeta+\sum_{\zeta=k+1}^{j-1}s_\zeta+s_j\\
S_j&>\sum_{\zeta=1}^{i-1}s_\zeta+s_i+\sum_{\zeta=i+1}^ks_\zeta\\
r_i+b_i&\ge r_k+b_k\\
S_j+r_i+b_i&>r_k+b_k+\sum_{\zeta=1}^{i-1}s_\zeta+s_i+\sum_{\zeta=i+1}^ks_\zeta\\
&\implies t_i^\epsilon>t_k\,\forall i<k<j\\
&\implies \exists t_i^\epsilon\ge t_k \forall i\le k\le j\\
&\implies \exists t_i\ge\mathrm{max}{t_i,\ldots,t_j}\\
&\forall t_k | k<i:\\
t_k&=r_k+b_k+\sum_{\zeta=1}^ks_\zeta\\
t_k^\epsilon&=r_k+b_k+\sum_{\zeta=1}^ks_\zeta\\
t_k^\epsilon&=t_k\\
t_k^\epsilon&=t_k\forall k<i\\
\mathrm{max}T\in\{t_1,\ldots,t_{i-1}&\implies\mathrm{max}T\in T^\epsilon\\
&\forall t_k | k>j:\\
t_k &= r_k+b_k+S_j+\sum_{\zeta=j+1}^ks_\zeta\\
t_K^\epsilon&=r_k+b_k+S_j+\sum_{\zeta=j+1}^ks_\zeta\\
t_k&=t_K^\epsilon | \forall k>j\\
\implies \exists t_m \in t^\epsilon &\ge t_k\forall t_k\in T | k>j\\
\exists t_i^\epsilon\in T^\epsilon\ge t_k\forall t_k\in T|i\le k\le j&\implies \exists t_{max}^\epsilon \ge t_k \ in T \forall k\ge i\\
\exists t_max^\epsilon \in T^\epsilon > t_k \in T \forall k<i&\implies\exists t_max\in T^\epsilon>t_k\forall t_k \in T\\
&implies \mathrm{max}T^\epsilon\ge\mathrm{max}T
\end{align*}




Discuss the ethical issues with parents deciding to send their kids to a camp where the counselor's name is Jason who is obsessed with the number 13.   

\item Given a binary string $x$ we write $x^k$ to denote $k$ copies of $x$ concatenated together.  We say that string $x'$ is a \textit{repeat} of $x$ if it is a prefix of $x^k$ for some number $k$.  So $x' = 1011011011$ is a repeat of $x=101$.   

We say that a string $s$ is a \textit{blending} of $x$ and $y$ if its symbols can be partitioned into two not necessarily contiguous subsequences $s_1$ and $s_2$ so that $s_1$ is a repeat of $x$ and $s_2$ is a repeat of $y$.  

For example, if $x = 101$ and $y = 00$ then $s= 100100010110011$ is an blending of $x$ and $y$, since characters $1,2,4,8,9,10,11,12,14,15$ form $1011011011$ which is a repeat of $x$ and the remaining characters, $3,5,6,7,13$ form $00000$ which is a repeat of $y$.  

Assume you are given three strings, $s$ (of length $n$) and $x$ and $y$.  Your job is to develop an efficient algorithm to determine if $s$ is a blending of $x$ and $y$.  

\item You are consulting for a manufacturer that ships to distributors all over the country.  For each of the next $n$ weeks they have a projected supply, $s_i$, of merchandise (measured in pounds) which has to be shipped.

Each week's supply can be shipped by one of two different companies. Company $A$ charges $r$ dollars per pound, Company $B$ charges a flat rate of $c$ dollars per week but requires a contract for four consecutive weeks of shipments.  You should create a polynomial time algorithm that chooses a shipping company for each of the $n$ weeks given the rules described so that your total shipping cost is minimized.

Example: \\
Suppose $r=1$ and $c=10$ and the projected supply in pounds is: $11,9,9,12,12,13,12,9,9,11$ then the optimal schedule would use company $A$ for the first three weeks, $B$ for the next four weeks and $A$ again for the last three weeks for a total cost of $98$.  

Note your algorithm must work in all cases but this example is designed to illustrate a couple of the trickier subcases.  




\end{enumerate}

\end{document}
